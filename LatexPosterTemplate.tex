% Based on "sig-alternate.tex" V2.0 May 2012
% This file should be compiled with V2.5 of "sig-alternate.cls" May 2012
%
% This example file demonstrates the use of the 'sig-alternate.cls'
% V2.5 LaTeX2e document class file. It is for those submitting
% articles to ACM Conference Proceedings WHO DO NOT WISH TO
% STRICTLY ADHERE TO THE SIGS (PUBS-BOARD-ENDORSED) STYLE.
% The 'sig-alternate.cls' file will produce a similar-looking,
% albeit, 'tighter' paper resulting in, invariably, fewer pages.
%
% ----------------------------------------------------------------------------------------------------------------
% This .tex file (and associated .cls V2.5) produces:
%       1) The Permission Statement
%       2) The Conference (location) Info information
%       3) The Copyright Line with ACM data
%       4) NO page numbers
%
% as against the acm_proc_article-sp.cls file which
% DOES NOT produce 1) thru' 3) above.
%
% Using 'sig-alternate.cls' you have control, however, from within
% the source .tex file, over both the CopyrightYear
% (defaulted to 200X) and the ACM Copyright Data
% (defaulted to X-XXXXX-XX-X/XX/XX).
% e.g.
% \CopyrightYear{2007} will cause 2007 to appear in the copyright line.
% \crdata{0-12345-67-8/90/12} will cause 0-12345-67-8/90/12 to appear in the copyright line.
%
% ---------------------------------------------------------------------------------------------------------------
% This .tex source is an example which *does* use
% the .bib file (from which the .bbl file % is produced).
% REMEMBER HOWEVER: After having produced the .bbl file,
% and prior to final submission, you *NEED* to 'insert'
% your .bbl file into your source .tex file so as to provide
% ONE 'self-contained' source file.
%
% ================= IF YOU HAVE QUESTIONS =======================
% Questions regarding the SIGS styles, SIGS policies and
% procedures, Conferences etc. should be sent to
% Adrienne Griscti (griscti@acm.org)
%
% Technical questions _only_ to
% Gerald Murray (murray@hq.acm.org)
% ===============================================================
%
% For tracking purposes - this is V2.0 - May 2012

\documentclass{acm_proc_article-sp}

\begin{document}

\title{My Fantastic Poster Title, based on {\ttlit ACM} SIG Proceedings Paper in LaTeX
Format}
%
% You need the command \numberofauthors to handle the 'placement
% and alignment' of the authors beneath the title.
%
% For aesthetic reasons, we recommend 'three authors at a time'
% i.e. three 'name/affiliation blocks' be placed beneath the title.
%
% NOTE: You are NOT restricted in how many 'rows' of
% "name/affiliations" may appear. We just ask that you restrict
% the number of 'columns' to three.
%
% Because of the available 'opening page real-estate'
% we ask you to refrain from putting more than six authors
% (two rows with three columns) beneath the article title.
% More than six makes the first-page appear very cluttered indeed.
%
% Use the \alignauthor commands to handle the names
% and affiliations for an 'aesthetic maximum' of six authors.
% Add names, affiliations, addresses for
% the seventh etc. author(s) as the argument for the
% \additionalauthors command.
% These 'additional authors' will be output/set for you
% without further effort on your part as the last section in
% the body of your article BEFORE References or any Appendices.

\numberofauthors{2} %  in this sample file, there are a *total*
% of EIGHT authors. SIX appear on the 'first-page' (for formatting
% reasons) and the remaining two appear in the \additionalauthors section.
%
\author{
% You can go ahead and credit any number of authors here,
% e.g. one 'row of three' or two rows (consisting of one row of three
% and a second row of one, two or three).
%
% The command \alignauthor (no curly braces needed) should
% precede each author name, affiliation/snail-mail address and
% e-mail address. Additionally, tag each line of
% affiliation/address with \affaddr, and tag the
% e-mail address with \email.
%
% 1st. author
\alignauthor
First Author (Only one author for SRC)\\
       \affaddr{1st author's affiliation}\\
       \affaddr{1st line of address}\\
       \affaddr{2nd line of address}\\
       \affaddr{telephone number, incl. country code}
       \email{Email address}
% 2nd. author
\alignauthor
Second Author\\
       \affaddr{2nd author's affiliation}\\
       \affaddr{1st line of address}\\
       \affaddr{2nd line of address}\\
       \affaddr{telephone number, incl. country code}
       \email{Email address}
% 3rd. author
%\alignauthor Lars Th{\o}rv{\"a}ld\titlenote{This author is the
%one who did all the really hard work.}\\
%       \affaddr{The Th{\o}rv{\"a}ld Group}\\
%       \affaddr{1 Th{\o}rv{\"a}ld Circle}\\
%       \affaddr{Hekla, Iceland}\\
%       \email{larst@affiliation.org}
%\and  % use '\and' if you need 'another row' of author names
%% 4th. author
%\alignauthor Lawrence P. Leipuner\\
%       \affaddr{Brookhaven Laboratories}\\
%       \affaddr{Brookhaven National Lab}\\
%       \affaddr{P.O. Box 5000}\\
%       \email{lleipuner@researchlabs.org}
%% 5th. author
%\alignauthor Sean Fogarty\\
%       \affaddr{NASA Ames Research Center}\\
%       \affaddr{Moffett Field}\\
%       \affaddr{California 94035}\\
%       \email{fogartys@amesres.org}
%% 6th. author
%\alignauthor Charles Palmer\\
%       \affaddr{Palmer Research Laboratories}\\
%       \affaddr{8600 Datapoint Drive}\\
%       \affaddr{San Antonio, Texas 78229}\\
%       \email{cpalmer@prl.com}
}
% There's nothing stopping you putting the seventh, eighth, etc.
% author on the opening page (as the 'third row') but we ask,
% for aesthetic reasons that you place these 'additional authors'
% in the \additional authors block, viz.
%\additionalauthors{Additional authors: John Smith (The Th{\o}rv{\"a}ld Group,
%email: {\texttt{jsmith@affiliation.org}}) and Julius P.~Kumquat
%(The Kumquat Consortium, email: {\texttt{jpkumquat@consortium.net}}).}
%\date{30 July 1999}
% Just remember to make sure that the TOTAL number of authors
% is the number that will appear on the first page PLUS the
% number that will appear in the \additionalauthors section.

\maketitle

\section{Introduction}
You are limited to a two-page abstract that describes your work plus one page for references only.  Your submission should tell the poster committee about your work and give us an idea of the content that will be on the poster you present at the conference.

The organization and text provided in this template is meant as a guide only.  For your work, the suggested text may belong in a different section.  Your may organize sections differently, but make sure that you clearly address each required component of the extended abstract.
In the introduction section, you should describe the open problem and motivation for solving the problem in sufficient details.   State the research question(s) you want to answer.

\section{Background and Related Work}

Describe specialized and relevant related background necessary to appreciate the work.  Include references to the literature where appropriate, and briefly explain where your work departs from that done by others.  Reference lists do not count towards the limit on the length of the abstract.
This is an example of a citation~\cite{Lamport:LaTeX}.

\section{Approach and Uniqueness}
Describe your approach in attacking the problem and how your approach is novel.  Your hypothesis may be here.

\subsection{Example subsection: Approach}
Describe your approach in attacking the problem and how your approach is novel.  Your hypothesis may be here.

\subsection{Example subsection: Evaluation}
To evaluate my approach to the problem, I designed the following experiment that answers the following research questions/evaluates my hypothesis.

\section{Results and Contributions}

Describe how the results of your work contribute to computer science and the significance of those results.

\subsection{Results}

Provide an overview of results.  Use tables or figures as appropriate.  Place Tables/Figures/Images in text as close to the reference as possible, typically at the top of a column (see Figure~\ref{fig:myfig}).  

\begin{figure}
\centering
\epsfig{file=process}
\caption{A sample black and white graphic (.pdf format).}
\label{fig:myfig}
\end{figure}

Captions should be numbered (e.g., ``Table 1'' or ``Figure 2''), please note that the word for Table and Figure are spelled out. Figures' captions should be centered beneath the image or picture, and Table captions should be centered above the table body.

\begin{table}
\centering
\caption{Table captions should be placed above the table.}
\begin{tabular}{|c|c|c|c|} 
\hline
\textbf{Graphics} & \textbf{Top} & \textbf{In-between} & \textbf{Bottom} \\ \hline
Tables & End & Last & First \\ \hline
Figures & Good & Similar & Very well \\ \hline
\end{tabular}
\end{table}


\subsection{Contributions}

The contributions of my work are:

\begin{itemize}
\item{Contribution 1.}
\item{Contribution 2.}
\end{itemize}

% The following two commands are all you need in the
% initial runs of your .tex file to
% produce the bibliography for the citations in your paper.
\bibliographystyle{abbrv}
\bibliography{sigproc}  % sigproc.bib is the name of the Bibliography in this case
% You must have a proper ".bib" file
%  and remember to run:
% latex bibtex latex latex
% to resolve all references

\balancecolumns
\end{document}
